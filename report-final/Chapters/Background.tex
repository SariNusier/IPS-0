\chapter{Background}

\section{Global Positioning System}
"GPS answers the questions 'What time, what position, and what velocity is it?' quickly, accurately, and inexpensively anywhere on the globe at any time" ~\cite{Hofmann}.  

In order to locate aircrafts and ships, the U.S. military developed a system called TRANSIT, the predecessor of the modern positioning system. It used six satellites at an altitude of 1,100 km, passing over one point on earth roughly every 90 minutes. The system worked by using the Doppler effect, the change of a wave's frequency for an observer relative to movement, to compute its location. This meant that users had to wait between "fixes" quite a long period, making the system unusable for real-time positioning, but enough for military applications ~\cite{Hofmann}. 

GPS solved this problem by ensuring that at least four satellites were always electronically visible. This allows for a receiver to compute its location using trilateration. The increase in accuracy and the ability to find one’s location in real-time, made the GPS suitable for civilian applications. 

Due to a higher complexity of indoor environments and the lack of line-of-sight transmissions between the receiver and the satellites, GPS is not suitable for indoor applications ~\cite{Gu}.

\section{Indoor Positioning System}
”In the last years a great deal of research has been conducted on developing methods and technologies for automatic location-sensing of people or devices” ~\cite{Flora}. Like the GPS, an IPS is a system that continuously and in real-time can determine the position of something or someone in a physical space ~\cite{Depsey}. Unlike GPS, IPS must be available for indoor environments ~\cite{Gu}. 

\subsection{General}
Hightower and Borriello ~\cite{Hightower} define three Location Sensing Techniques: Triangulation, Scene Analysis and Proximity. 

Triangulation can be made through lateration or angulation. For the lateration approach, distance from the device to three or four points (non-collinear/non-coplanar) must be measured. Using a direct form of measurement is the most accurate and straightforward solution, but very hard to implement in an automated system. Another approach is the use of Time-of-
Flight, the time taken to reach a reference point, knowing the velocity. Time synchronisation is an important factor to consider when using radio waves to calculate distance. Because indoor environments are usually small, the smallest time discrepancy can greatly affect accuracy. Another issue is managing duplicate pulses. That is, same pulse arriving multiple times, due to reflections. Attenuation, the signal strength decrease over distance, can also be used as a measurement method. “In environments with many obstructions such as an indoor office space, measuring distance using attenuation is usually less accurate than time-of-flight” ~\cite{Hightower}.

Proximity location sensing determines if a device is near a known point. This can be detected through physical contact or using the limited ranges of Radio transceivers, such as Wi-Fi access points. Another way of determining proximity can be through user interaction. For example in a museum, if each exhibit has a QR code or an NFC tag for the user to interact with, a scan of the two can imply that the user is viewing the exhibit and is near it.
\subsection{Current Approaches}

\noindent \textbf{Active Badge}

Developed by AT\&T Cambridge between 1989 and 1992, the active badge system is one of the first IPS created ~\cite{Want},~\cite{ActiveBadge}. The system works by attaching an “active badge” to people inside the building. Pulse-width modulated infrared signals are emitted by the badge every 15 seconds, encoding the unique identifier of the badge. The signals are detected by sensors placed around the buildings. A central system processes the data received from the sensors about “sightings” and makes the location information available for users. Both badges and sensors are relatively cheap, the costs of the system mainly coming from the wired connections that must be made between the sensors and the central processing system. 

Using modern technology, the costs can be drastically reduced. Wired connections can be replaced by cheap wireless modules. Dedicated hardware for badges can be replaced by IR enabled smartphones. This approach offers “room level” accuracy, which is in line with the requirements of the project. One of the problems with this approach comes from the need of having line-of-sight between the badges and sensors. Another problem is that the IR signal can be influenced by other light sources, making it difficult for sensors to pick up the signals. The ability to read data in parallel from multiple devices would increase the complexity and costs of the sensors. This problem would arise if multiple devices are sending a stream of bits simultaneously. Shortening the duration of each transmission can reduce the likelihood of such collisions, but will also drop the bitrate, reducing the number of unique badges that can be generated. Another downside is the lack of IR blasters in smartphones. Even though the Android SDK offers infrared capabilities, the official android compatibility guide does not make any reference to infrared, whereas bluetooth compatibility is mandatory ~\cite{AndroidCompat}.

Other IR based systems, such as Firefly ~\cite{Firefly1},~\cite{Firefly2} or Optotrak ~\cite{Optotrak}, are available but are too expensive and are usually used in applications where a high degree of accuracy is required. 

\medskip
\noindent \textbf{Active Bat}

This approach was also developed by AT\&T Cambridge ~\cite{ActiveBat} and is similar to the Active Badge system. Each person must wear a tag emitting short ultrasound pulses. Sensors placed on the ceiling calculate the distance to the source using the Time-Of-Arrival. The measurements are then used to triangulate the location of the source. The problem of this approach comes mainly from the high costs and the time and effort required to install the sensors. The accuracy of the system is also influenced by reflections and obstacles between tags and sensors. Cricket ~\cite{Priyantha1}, ~\cite{Priyantha2} is another system using a similar approach.

\medskip
\noindent \textbf{Wi-Fi triangulation approach}

Radar ~\cite{Bahl} and Ekahau ~\cite{Ekahau} are two similar approaches to using the already present Wi-Fi infrastructures. Wi-Fi enabled devices are located using the triangulation technique, where distances are measured based on received signal strength (RSSI). One advantage of these systems is that they use the current infrastructure, and no additional hardware is required. Even though they are cheap and easy to implement, using the Wi-Fi RSSI in triangulation has been proven inaccurate, especially when considering the user’s influence on signal strength. 

\medskip
\noindent \textbf{Wi-Fi Fingerprinting and probabilistic approach}

Compass \cite{King} is an experimental system that uses the Wi-Fi infrastructure to determine location. Unlike the previous examples, compass takes into account the user’s orientation, by using digital compasses. Because the designers tested the system by tracking only one user at a time, it is impossible to determine the scalability of such an approach.

Kaemarungsi and Krishnamurthy ~\cite{Kaemarungsi} review the use of WLAN Location Fingerprinting as a way of locating a device in physical space. Small, A. Smailagic, and D. P. Siewiorek ~\cite{Small} found that the distribution of RSS in dB from Wi-Fi access points follows a Gaussian Distribution. Kaemarungsi and Krishnamurthy ~\cite{Kaemarungsi} argue that this might be due to the lack of user presence when the measurements were taken. They have found that, when a user is present, the distribution was non-Gaussian and asymmetric. They conclude that it is a better approach to record the distribution of RSS rather than calculate and use the mean value. 
Battiti, M. Brunato ~\cite{Battiti} use Statistical Learning to determine the location of devices. The first use Bayesian Approach, WKNN and Support Vector machines to solve the location positioning as a regression problem. Another approach taken, similar to the one used in this project, was to consider positioning as a classification problem, labelling each room in the building and using a learning set for each room. Out of 257 measurements, they found the following results:
\begin{itemize}
\item SVM: classifier: 8 errors, regression: 20 errors.
\item Multilayer perceptron: classifier: 34 errors, regression: 31 errors.
\item WKNN: regression: 15 errors
\item Bayesian regression: 33 errors.
\end{itemize}
These results will be used to compare with the end results of this project.

\section{Machine Learning}

"The field of pattern recognition is concerned with the automatic discovery of regularities in data through the use of computer algorithms and with the use of these regularities to take actions such as classifying the data into different categories" ~\cite{Bishop}.
One part of machine learning is concerned with supervised learning, in which the training data is a set of input vectors, given along with their corresponding target vectors. When the goal is to associate each input vector to one of a finite set of categories, it is called a classification problem. Understanding the mathematics behind the classifiers used in the project is not needed for implementation, as the classifiers are implemented using the Weka Library ~\cite{Weka}, but it might help in choosing which classifier to be used when predicting the location based on RSS. Even though this project will use a statistical approach to picking the best classifiers, limited mathematical definitions of some classifiers will be included below. For more details, a great source is Bishop ~\cite{Bishop}.

\subsection{Naive Bayes classifiers}
This is one of the most popular approach to classifiers. Murphy ~\cite{Murphy} states that,even though usually false, the assumption made by Naive Bayes is that all the features are conditionally independent given the class label:
$$ p(x|y = c) = \prod\limits_{i=1}^D p(x_i|y = c) $$

As we have seen in the previous section (2.2.2), the data containing the RSS is usually Gaussian distributed, especially when a user is not present as the measurements are not taken. In this case, we get:
$$ p(x|y = c,\theta_c) = \prod\limits_{i=1}^D \mathcal{N}(x_i|\mu_{ic} = \sigma_{ic}) $$ and must estimate $C \times D$ separate Gaussian parameters, $\mu_{ic}$, $\sigma_{ic}$, where $C$ is the set of class labels and $D$ is the set of features, as real numbers [9].

\subsection{Bayesian Networks}
When a large number of variables are used, such as in our case, many Access Points, it is not efficient to represent the dependencies between each variable. To solve this, a graph can be built to represent those dependencies, by describing complex joint distributions using local distributions. In other words, we can see how variables interact locally, and chain these interactions to see the global interaction.
Consider an arbitrary joint distribution $p(a,b,c)$ over three variables $a$, $b$, and $c$.
By applying the product rule of probability, $p(X,Y) = p(Y|X)p(X)$ the joint distribution will become $$p(a,b,c) = p(c|a,b)p(a,b)$$
A second application of the product rule, this time to the second term on the righthand gives $$p(a,b,c) = p(c|a, b)p(b|a)p(a)$$
We now represent the right side of the equation as a graph by introducing a node for each variable and associate each node with the corresponding conditional distribution. That is, $c$ is dependent on $a$ and $b$, $b$ is dependant on $a$ and $a$ is independent. A general example can be considered for the joint distribution over $K$ variables given by $p(x_1,x_2,...,x_K)$. As seen before, the joint distribution can be written as: $p(x_1,...,x_K) = p(x_K|x_1,...,x_{K-1})...p(x_2|x_1)p(x_1)$

\section{Orienteering}
Golden, Levy and Vohra ~\cite{Golden}, define the orienteering problem as follows: Given $n$ nodes, each node with a score greater or equal than 0, a route of maximum score must be found, which is no longer than a predefined value. The definition of the problem fits well with the requirements of a planned route through a museum. That is, each visitor should select which exhibits they would like to visit, giving a score on how much they would like to visit each exhibit. Based on the estimated time of visit for each exhibit, the route generated must take no longer than the maximum time that they have allocated for the visit.

Benton, Coles and Coles ~\cite{Coles} have considered temporal planning where the objectives are not only to minimise makespan, but to also achieve other goals, in a specified time frame. The planner created, OPTIC ~\cite{OPTIC}, will be used to solve the orienteering problem in our project.







