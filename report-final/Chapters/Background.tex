\chapter{Background}

\section{Global Positioning System}
"GPS answers the questions 'What time, what position, and what velocity is it?' quickly, accurately, and inexpensively anywhere on the globe at any time" (Hofmann-Wellenhof, Lichtenegger and Collins, 1997).  

In order to locate aircrafts and ships, the U.S. military developed a system called TRANSIT, the predecessor of the modern positioning system. It used six satellites at an altitude of 1,100 km, passing over one point on earth roughly every 90 minutes. The system worked by using the Doppler effect, the change of a wave's frequency for an observer relative to movement, to compute its location. This meant that users had to wait between "fixes" quite a long period, making the system unusable for real-time positioning, but enough for military applications (Hofmann-Wellenhof, Lichtenegger and Collins, 1997). 

GPS solved this problem by ensuring that at least four satellites were always electronically visible. This allows for a receiver to compute its location using trilateration. The increase in accuracy and the ability to find one’s location in real-time, made the GPS suitable for civilian applications. 

Due to a higher complexity of indoor environments and the lack of line-of-sight transmissions between the receiver and the satellites, GPS is not suitable for indoor applications (Gu, Lo and Niemegeers, 2009).

\section{Indoor Positioning System}
"In the last years a great deal of research has been conducted on developing methods and technologies for automatic location-sensing of people or devices" [3]. 

Like the GPS, an IPS is a system that continuously and in real-time can determine the position of something or someone in a physical space [2]. Unlike GPS, IPS must available for indoor locations. Gu, Lo and Niemegeers [2] 
The first classification method they use is based on whether the systems uses the current infrastructure or not. A network-based approach will require no additional hardware, considerably reducing the cost. On the other hand, a non-network-based approach might offer higher degrees of accuracy, as the infrastructure must be designed and set-up for the specific application. Another classification method used is based on the medium used to determine location: Infrared (IR) is a common system because of the high availability of IR and its presence in everyday consumer devices. Roy et al. defines an Active Badge system which uses diffused IR to send signals every 15 seconds [3],[4]. Receivers are placed in each room, collecting the data and using it identify the location of the different emitters [3],[4]. This is a possible solution for the museum application, as the main focus of the system is to identify the room or section that the visitor is currently in, high position . This can be also applied to specific exhibitions, where receivers can be placed, reading the signal from the nearby phones. There are multiple downsides of this approach. Mainly, it is hard to read data that comes in parallel from multiple devices. This problem would appear if there are multiple devices sending a stream of bits simultaneously. Shortening the duration of each transmission can reduce the likelihood of collision, but will also drop the bitrate, reducing the number of unique badges that can be generated. Another downside is the lack of IR blasters in smartphones. Even though the Android SDK offers infrared capabilities, the official android compatibility guide does not make any reference to infrared, whereas bluetooth compatibility is mandatory [5]. The usage of radio is also very common in indoor positioning systems. These can range from using RFID tags to identify people as they enter rooms, to bluetooth transmitters, but also the use of WLAN networks that are already installed on sight. As mentioned above, using the already installed WLAN system represents a network-based approach, which is cost effective and time effective, whereas the non-network-based approach offers a greater deal of control [2]. This project will keep the goal of creating a positioning system which is relatively independent of the means used for measuring distance. But for convenience, it will use bluetooth modules to measure signal strength, and therefore, distance.

\section{Machine Learning}

"The field of pattern recognition is concerned with the automatic discovery of regularities in data through the use of computer algorithms and with the use of these regularities to take actions such as classifying the data into different categories". (Bishop, 2006)
One part of machine learning is concerned with supervised learning, in which the training data is a set of input vectors, given along with their corresponding target vectors. When the goal is to associate each input vector to one of a finite set of categories, it is called a classification problem.

