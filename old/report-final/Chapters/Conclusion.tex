\chapter{Conclusion and Future Work}

\section{Conclusion}
The difficulty of implementing a project without a team comes from the lack of diversity in skills that different team members would bring. This is also the best thing about working alone, as it requires people to learn new skills in order to reach their goals. 

During the development of this project, I had the chance to learn many things, from time and project management to new technologies and how to find and choose the right tools. I have learnt how to program using JavaScript and how to use NodeJS and other frameworks available to create a reliable and easy to develop backend. I have learnt about NoSql databases and how to model and query data efficiently using them. 
I have also improved my skills in Java and Android development by creating two mobile applications and a Java server. I have learnt about Machine Learning and its capabilities and limitations. I have learnt about Planning and PDDL, how to create a domain file and how to generate a problem file. The knowledge and skills gained through the individual project will help me throughout my career in Computer Science. It has been a difficult, but rewarding experience. 

The software developed in the project is far from complete. It was developed just as a first step towards a bigger project. It was used to prove the reliability of the idea and to provide a system which can be later extended and improved. It has been shown that, it is possible to determine, with high levels of precision, the indoor location of a Wi-Fi enabled device by using the already existing infrastructure. This means that no dedicated hardware is needed, greatly reducing the costs of implementing such a system in a real life scenario. It is also a great subject for future research into how precision can be further increased and, later on, accuracy. 
 
The next section will list some possible improvements that can be made to the existing software.

\section{Future Work}
\subsection{Security and Data collection}
The most important features that must be added to the next version are security related. Login methods must be implemented so that only the owners of a building can have access to the data stored. The idea behind this project doesn’t focus only on the usefulness for the visitors, but also for the hosts. It can be used by the management of museums to identify how people visit the museum, where they spend the most time and receive reviews at the end of the visits. It would help museums advertise their exhibits and improve their services. If such a system is to be developed, login methods must be implemented for visitors too. This would mean that issues concerning privacy will be raised. It is therefore important that such issues, which have not been addressed yet, will be addressed before the system becomes available to the public. 

\subsection{Graphics}
Another issue with the current implementation is the lack of a quality GUI. The interface is quite basic, just enough for the system to be used. It is important that a more practical and attractive interface is designed. This will help attract more users and improve the overall quality of the software. Graphics should also be implemented for displaying the indoor maps. It would allow hosts to easily create and manage the layout of their buildings and display to the visitors an interactive map, showing their locations, exhibits that can be viewed, more information about the exhibits that they are viewing and suggested paths that they can take for a better experience. 

\subsection{Optimisation}
Another weak point of the current implementation is its lack of optimisation. It is very likely that the current version would not be able to hold a large user base. The tools used in the implementation were not optimised for scalability. This means that, some modules, such as the planning and positioning system, must be modified to allow the handling of multiple simultaneous requests. Both android applications must also be optimised. Most importantly, the HTTP requests must be handled outside the UI thread. Caching should also be implemented to reduce the amount of data loaded from the internet. One approach can be to implement a broadcast system that checks whether data has been changed and updates the local cache when needed.

\subsection{Proximity interactions}
Even though the current implementation can identify when a visitor is viewing an exhibit, it does not offer any additional information. The app can be developed to show details about the exhibit being viewed. Cheaper methods for detecting viewers can also be added, such as QR Codes or NFC Tags. 

